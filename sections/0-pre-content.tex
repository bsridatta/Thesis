\newpage
% \thispagestyle{plain}
% ~\\
% \vfill
% { \setstretch{1.1}
%     \subsection*{Authors}
%     Sri Datta Budaraju <budaraju@kth.se>\\
%     School of Electrical Engineering and Computer Science\\
%     KTH Royal Institute of Technology
    
%     \subsection*{Place for Project}
%     Stockholm, Sweden\\
%     Stuttgart, Germany

%     \subsection*{Examiner}
%     Danica Kragic Jensfelt\\
%     Stockholm, Sweden\\
%     KTH Royal Institute of Technology
    
%     \subsection*{Supervisor }
%     Hedvig Kjellström\\
%     Stockholm, Sweden\\
%     KTH Royal Institute of Technology
    
%     \subsection*{Supervisor - Host}
%     Arij Bouazizi\\
%     Stuttgart, Germany\\
%     Mercedes-Benz AG,  Research and Development
%     ~
% }

% TODO -- uncomment
\newpage
\thispagestyle{plain}
%%%%%%%%%%%%%%%%%%%%%%%%%%%%%%%%%%%%
%%  The English abstract          %%
%%%%%%%%%%%%%%%%%%%%%%%%%%%%%%%%%%%%
\chapter*{Abstract}
%%%%%%%%%%%%%%%%%%%%%%%%%%%%%%%%%%%%

The thesis proposes an unsupervised representation learning method to predict 3D human pose from a 2D skeleton via a VAE-GAN (Variational Autoencoder - Generative Adversarial Network) hybrid network. The method learns to lift poses from 2D to 3D using self-supervision and adversarial learning techniques. The method does not use images, heatmaps, 3D pose annotations, paired/unpaired 2D-to-3D skeletons, 3D priors, synthetic 2D skeletons, multi-view or temporal information in any shape or form. The 2D skeleton input is taken by a VAE that encodes it in a latent space and then decodes that latent representation to a 3D pose. The 3D pose is then reprojected to 2D for a constrained, self-supervised optimization using the input 2D pose. Parallelly, the 3D pose is also randomly rotated and reprojected to 2D to generate a 'novel' 2D view for unconstrained adversarial optimization using a discriminator network. The combination of the optimizations of the original and the novel 2D views of the predicted 3D pose results in a 'realistic' 3D pose generation. The thesis shows that the encoding and decoding process of the VAE addresses the major challenge of erroneous and incomplete skeletons from 2D detection networks as inputs and that the variance of the VAE can be altered to get various plausible 3D poses for a given 2D input. Additionally, the latent representation could be used for cross-modal training and many downstream applications. The results on Human3.6M datasets outperform previous unsupervised approaches with less model complexity while addressing more hurdles in scaling the task to the real world.


\subsection*{Keywords}
Computer Vision, Projective Geometry, Deep Learning, Unsupervised Learning, 3D Human Pose Estimation, GAN, AutoEncoder, Hybrid Generative Model, Self-Supervision
% uncomment

% TODO -- uncomment
\newpage
\thispagestyle{plain}
%%%%%%%%%%%%%%%%%%%%%%%%%%%%%%%%%%%%
%%   The Swedish abstract         %%
%%%%%%%%%%%%%%%%%%%%%%%%%%%%%%%%%%%%
\chapter*{Sammanfattning} 
%%%%%%%%%%%%%%%%%%%%%%%%%%%%%%%%%%%%

Denna avhandling föreslår en oövervakad representation lärande [unsupervised representational learning] metod för att förutspå en 3D-pose från ett 2D-skelett via ett VAE-GAN (Variationella Autokodare-Generativt Kontroversiellt Nätverk)[Variational Autoencoder - Generative Adversarial Network] hybrid neural nätverk. Metoden lär sig att utveckla poser från 2D till 3D genom att använda övervakning [supervision] och Kontroversiellt [adversarial] inlärningstekniker. Metoden använder sig inte av bilder, värmekartor, 3D-pose annotationer, parade/oparade 2D-till-3D skelett, 3D föregående [3D priors], syntetiska 2D-skelett, "multi-view", eller tidsinformation överhuvudtaget. 2D-skelett indatat är taget från ett VAE som kodar den i ett latent utrymme [latent space] och sedan avkodar det latent representationen till ett 3D-pose. 3D-posen är sedan åter projicerad till 2D för att genomgå ett begränsad, självövervakad [self-supervised] optimering med hjälp av det tvådimensionella posen. Parallellt roteras dessutom 3D-posen slumpmässigt och åter projiceras till 2D för att generera ett nytt 2D-vy för obegränsat kontroversiellt [adversarial] optimering med hjälp av ett diskriminatorn [discriminator] nätverk. Kombinationen av optimeringarna av det originala och det nya 2D vyna av det förutsedda 3D-posen resulterar i en realistisk 3D-pose generering. Denna avhandling visar att kodnings- och avkodningsprocessen av VAE adresserar den stora utmaningen av felaktiga och ofullständiga skelett från 2D-detekteringsnätverk som indata och att variansen av VAE kan bli modifierat för att få flera troliga 3D-poser för ett givet 2D-indata. Dessutom, kan det latenta representationen användas för crossmodal [cross-modal] träning och flera nedströmsapplikationer. Resultaten av Human3.6M datamängder presterar bättre än tidigare oövervakad [unsupervised] metoder med mindre modellkomplexitet samtidigt som det adresserar flera hinder att skala uppgiften till det riktiga världen.

% \subsection*{Nyckelord}
% Computer Vision, Projective Geometry, Deep Learning, Unsupervised Learning, 3D Human Pose Estimation, GAN, AutoEncoder, Hybrid Generative Model, Self-Supervision

\newpage
\thispagestyle{plain}
\chapter*{Acknowledgements}

I would like to express my sincere gratitude to my supervisor at KTH, Hedvig Kjellström, and my industrial supervisor, Arij Bouazizi for their patient guidance, encouragement, and constructive critique of this research work. I am thankful to my examiner, Danica Kragic Jensfelt for her feedback and evaluation of this thesis. I would like to thank Mercedes Benz for hosting my thesis and providing me with the required infrastructure. I would like to express my appreciation to Ulrich Kressel and Julian Wiederer for welcoming me to their research group. I would like to acknowledge the helpful feedback, support, and company from my thesis group at KTH and would like to thank Ruibo Tu, Jade Cock, and Oscar Örnberg for their immense help. I am thankful to my friends Nik Vaessen and Vivek Chalumuri for their helpful discussions, comments, and advice throughout the whole process. Finally, I am grateful to my family for their lifelong support. This thesis was carried out during a period of unprecedented uncertainty and was only possible as a result of the incredible patience and empathy shown by everyone involved in the process.   

\newpage

\chapter*{Acronyms}

\begin{acronym}[TBD]
    \acro{ann}[ANN]{Artificial Neural Network}
    \acro{ar/vr}[AR/VR]{Augmented Reality/Virtual Reality}
    \acro{hpe}[HPE]{Human Pose Estimation}
    \acro{kld}[KLD]{Kullback–Leibler Divergence}
    \acro{l1}[L1]{Least Absolute Deviations}
    \acro{mmvae}[MMVAE]{Mixture-of-Experts Multimodal Variational Auto-Enocder}
    \acro{mpjpe}[MPJPE]{Mean Per Joint Position Estimate}
    \acro{pjpe}[PJPE]{Per Joint Position Estimate}
    \acro{mse}[MSE]{Mean Squared Error}
    \acro{mvae}[MVAE]{Multimodal Variational Auto-Enocder}
    \acro{mocap}[MoCap]{Motion Capture}
    \acro{nrsfm}[NRSfM]{Non-Rigid Structure from Motion}
    \acro{pov}[POV]{Point of View}
    \acro{rgb}[RGB]{Red Green Blue}
    \acro{sota}[SOTA]{State-of-The-Art}
    \acro{umap}[UMAP]{Uniform Manifold Approximation and Projection}
    \acro{vae}[VAE]{Variational Auto-Encoder}
    \acro{gan}[GAN]{Generative Adversarial Network}
    \acro{wgan}[WGAN]{Wasserstein Generative Adversarial Network}
    \acro{wgangp}[WGAN-GP]{\ac{wgan} with Gradient Penalty}
    \acro{jsd}[JS-divergence]{Jensen–Shannon Divergence}
    \acro{emd}[EM distance]{Earth Mover's Distance}
    \acro{bvae}[$\boldsymbol{\beta}$-VAE]{Beta Variational Auto-Encoder}
    \acro{fc}[FC]{Fully Connected}
    \acro{relu}[ReLU]{Rectified Linear Unit}
    
\end{acronym}



\newpage

\etocdepthtag.toc{mtchapter}
\etocsettagdepth{mtchapter}{subsection}
\etocsettagdepth{mtappendix}{none}
\thispagestyle{plain}
\tableofcontents

\newpage


